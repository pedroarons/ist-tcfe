\section{Theoretical Analysis}
\label{sec:analysis}
\paragraph{}

\par In this section, the circuit is analyzed in theory through Octave calculations. The BPF circuit shown in the introduction basically consists of a high pass filter in series with a low pass filter and a signal amplifier. To be able to make calculations, the OpAmp characteristics are assumed to be ideal, that means its impedance is infinite for the input and null for output.
\par With the values for the components given in table 1, it is possible to calculate the following values:

\begin{table}[H]
    \centering
    \begin{tabular}{|c|c|}
    \hline
        \input{../mat/TA.tex}
    \end{tabular}
    \caption{Values for Gain, Input and Output impedances calculated thgrouh Octave}
    \label{TA}
\end{table}

\par Finally the frequency response $V_0(f)/V_i(f)$ is ploted for the gain:

\begin{figure}[H]
	\includegraphics[width=0.5\linewidth]{gain.eps}
	\centering
	\caption{Gain plot - $\frac{V_o(f)}{V_i(f)}$}
	\label{gain}
\end{figure}

\par And for the phase:

\begin{figure}[H]
	\includegraphics[width=0.5\linewidth]{phase.eps}
	\centering
	\caption{Phase plot}
	\label{pha}
\end{figure}
