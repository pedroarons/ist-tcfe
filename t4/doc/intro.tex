\section{Introduction}
\label{sec:introduction}
\paragraph{}
\par In this assignment we were tasked with the creation of an audio amplifier circuit and its analysis, both theoretically and through NGSpice.
\par An audio amplifier is a circuit configuration in which an imput signal is received, amplified and sent to a speaker. Our input signal will have a voltage of 10 mV and will be outputted into an 8 $\Omega$ speaker.
\par This circuit comprises 2 stages: a gain stage, where the signal is significantly amplifed with the cost of also increasing its impedance. This stage's main component is a NPN transistor; and an output stage where, in order to bridge the previous problem, reduces the impedance keeping the same signal amplitude. This stage has a PNP transistor as its primary component. This fulfills the goal of having an increased signal without any noticeable gain loss.
\par However, there can be setbacks in these situations which is why the quality of the designed circuit has a figure of merit that follows the equation below:
\[Merit=\frac{Gain \cdot Bandwidth}{Cost \cdot LowerCutOffFrequency}\]
\par In the laboratory description, the following general design of the circuit was given:

\begin{figure}[H]
    \includegraphics[width=0.5\linewidth]{circuit.pdf}
    \centering
    \caption{Studied Circuit}
    \label{circuit}
\end{figure}


\par The gain and output stages were added to this information, creating the final circuit shown below: 

\begin{figure}[H]
    \includegraphics[width=0.5\linewidth]{Lab4.pdf}
    \centering
    \caption{Studied Circuit}
    \label{circuit}
\end{figure}

\par In Section~\ref{sec:analysis}, a theoretical analysis of the circuit is
presented. In Section~\ref{sec:simulation}, the circuit is analysed by
simulation using NGSpice, with its results being compared to the theoretical results obtained in
Section~\ref{sec:analysis} in the Section~\ref{sec:conclusion}, while also outlining in this section the conclusions of this study.	
