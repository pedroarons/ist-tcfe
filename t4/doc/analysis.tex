\section{Theoretical Analysis}
\label{sec:analysis}
\paragraph{}

\par As previously mentioned, this circuit is characterized by a Gain Stage and an Output Stage.
\par In this section, we were able to obtain the desired values using the hand calculations provided by the professor and resorting to the Operating Point Analysis.
\par The values for gain and input and output impedances is shown in both tables below, respectively:

\begin{table}[H]
    \centering
    \begin{tabular}{|c|c|}
    \hline
        \input{../mat/GS.tex}
    \end{tabular}
    \caption{Gain Stage}
    \label{table4a}
\end{table}

\begin{table}[H]
    \centering
    \begin{tabular}{|c|c|}
    \hline
        \input{../mat/OS.tex}
    \end{tabular}
    \caption{Output Stage}
    \label{table4a}
\end{table}

\par The values for gain and input and output impedances for the full circuit is shown in the following table:

\begin{table}[H]
    \centering
    \begin{tabular}{|c|c|}
    \hline
        \input{../mat/TOTAL.tex}
    \end{tabular}
    \caption{Full circuit}
    \label{table4a}
\end{table}



\par It is possible to plot the frequency response as well by computing the gain $\frac{V_o(f)}{V_i(f)}$, as one can view in the next graph:

\begin{figure}[H]
	\includegraphics[width=0.5\linewidth]{gain.eps}
	\centering
	\caption{Gain plot - $\frac{V_o(f)}{V_i(f)}$}
	\label{pha}
\end{figure}


